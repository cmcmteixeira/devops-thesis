\chapter{ Conclusions } \label{chap:conclusion}
    \section{Contributions}
    As of this moment, DevOps is still evolving and as more people join the discussion more perspectives and experiences will become part of the movement. Regardless, we believe that the initial work made in this thesis towards a more structured way of presenting and talking about DevOps will help reduce the barriers for new adopters and help the movement grow by providing a common dialect for discussing DevOps. This work can also be seen as a starting point for further investigation.



    \section{Future Work}
      \subsection{Specializing the identified patterns}
      While pursuing the study of DevOps we kept a high level of granularity with the objective of capturing a wider view of the movement and its practices. We believe that having done this work, each pattern should be further analyzed in a more precise way.
      \subsection{DevOps monitoring}
      As refered before, the DevOps movement is still evolving. We believed that, by monitoring the DevOps movement it will be possible to identify more practices being employed by companies and individuals.
      \subsection{Further validation}
      The initial results observed hint that DevOps and its practices can bring benefits to both companies and individuals. In this thesis we were not able to fully validate all of the identified patterns and we also did not observe long term effects of this developments in both teams and companies. Validation was also done with only one team and one company which is insufficient to prove that those benefits can be reproduced in other teams or organizations. 

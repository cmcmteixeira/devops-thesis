\chapter{Introduction} \label{chap:introduction}
	Altough DevOps ideas are not recent (DevOps ideals can be traced back as far as 2008) academic and peer reviewed literature about the subject is scarce.

	Having notice that, we aim, in this thesis, to study DevOps and the way it is being applied near the industry. With this study, we aim to enable those that want to adopt DevOps and those who want to study DevOps with a starting point on which to build.






	\section{Context} \label{chap:introduction:sec:context}
	Developing and maintaining/operating software are often disjoint tasks and responsabilities. This pattern has been observed in several organizations and project management techniques. Nevertheless, this was not always the case, and, in the beginning, the same person that developed the software was also the person that operated it \cite{Loukides2012}.

	As a result of this separation, two separated departments were born. This two departments, Development and Operations, would not be able to efficiently articulate preventing businesses from beeing as efficient as they could be.

	Devops appeared within this context with the objective of solving this lack of articulation between those that developed and those that maintained/operated.





	\section{Problem} \label{chap:introduction:sec:problem}

	The DevOps movement spans throughout a vast set of areas and tries to change both techniques and culture. This variety means, nonetheless that Devops can offer an integrated approach that takes into consideration several aspects of the software pipeline but, at the same time, makes it hard to fully study and understand.

	Looking at the current state of DevOps, we can see that the movement popularity was and is rising. We can find, with a simple search, numerous blog posts made both by individuals and companies about their experiences and opinions regarding Devops an its adoption. Not only that, but there are also a growing number of "Devops ready" tools that promise to simplify and empower companies with the benefits of Devops.

	Contrastingly, a simmilar search on a search engines like "Engineering Village" or "Scopus" will yield a much smaller list of results.

	This lack of academic literature and study means that Devops understandment is still mostly based on opinions and personal experiences. Consequently, adopting Devops is not trivial.




	\section{Motivation} \label{chap:introduction:sec:motivation}

	Devops represents a new way to look at the entire software pipeline. From development to delivery and maintenance, Devops represents an advantage for both teams and businesses by creating a more efficient, agile and collaborative way of working. From a more business centric perspective, DevOps also reduces the software time to market which provides a competitive advantage for those that are able to practice it.

	With that in mind, we believed that studying Devops, its values and common practices, will allow for a broader adoption and further strengthening of the movement.




	\section{Goals}
	The main goal of this thesis is to increase the existent knowledge regarding DevOps. With this increment, teams and companies that want to adopt DevOps will be able to better understand common pitfalls and solutions related with DevOps. At the same time, further studies of the DevOps movement will be able to build upon this study by extending the current concepts or by having a base from which to search new ones.


  \section{Contributions} \label{chap:introduction:sec:contributions}
	Taking into account the goals defined for this thesis, we defined the contributions of this thesis to be:
    	\begin{itemize}
    		\item{A comprehensive description of the current state of the art regarding the DevOps movement.}
    		\item{Set of patterns and good practices concerning DevOps.}
    	\end{itemize}

  \section{Outline} \label{chap:introduction:sec:outline}
    	\TODO{This report was made as preparation for the dissertation.}

    	% In it there is a Introduction \ref{chap:introduction}, followed by the state of the art \ref{chap:stateoftheart} regarding the reach and size of the Internet \ref{chap:stateoftheart:sec:internet}, cloud computing \ref{chap:stateoftheart:sec:cloud} and DevOps \ref{chap:stateoftheart:sec:devops}.
    	% Section \ref{chap:towardsdevops} defines both the problem \ref{chap:towardsdevops:sec:problem}, the solution \ref{chap:towardsdevops:sec:solutionperspective}, and the methodology \ref{chap:towardsdevops:sec:methodology}.
    	% Then, the method for validating the obtained results is described in \ref{chap:validation}.
			%
    	% Lastly, the Conclusion is presented and the planning for the future work is presented \ref{chap:conclusion}.

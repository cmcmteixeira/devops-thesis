\chapter{Introduction} \label{chap:introduction}
The name DevOps comes from joining the words 'Development' and 'Operations'. With some of the original work that triggered the rise of the DevOps movement being traced back as far as 2008, we have been able to find multiple accounts and histories of personal opinions regarding the subject. This information has not, however, been the subject of scientific and academic analysis meaning that DevOps is still surrounded by uncertainty.

		\section{Context} \label{chap:introduction:sec:context}
		Developing and maintaining/operating software are often seen as disjoint tasks and responsibilities. This pattern has been observed in several organizations like the ones described in \textit{Agile infrastructure and operations: How infra-gile are you?} \cite{Debois2008} and project management techniques like Waterfall \cite{Royce1970}. Nevertheless, this was not always the case, and, in the beginning, the same person that developed the software was also the person that operated it \cite{Loukides2012}.\\
		As a result of this separation, two separated departments often exist. This two departments, Development and Operations, are usually not able to efficiently articulate which causes friction between the two as well as a bottle neck for businesses.

		\section{Problem} \label{chap:introduction:sec:problem}

		The DevOps movement spans throughout a vast set of areas and tries to change both the technical and cultural aspects related with software development and with the software operations.\\
		Looking at the current state of DevOps, we can see that the movement popularity was and is rising. We can find, with a simple Google search, numerous blog posts about experiences and opinions regarding Devops and its adoption. Not only that, but there are also a growing number of \textit{Devops ready} tools that promise to simplify and empower companies with the benefits of Devops.\\
		Contrastingly, a similar search on \textit{Scopus} will yield close to 200 results which is a much smaller list when compared with other terms like \textit{SCRUM} (more than 2000 results) or \textit{Waterfall} (more than 700 results).\\
		This lack of academic literature and study means that Devops understanding is still mostly based on opinions and personal experiences rather than scientific, peer reviewed literature. Consequently, adopting and practicing DevOps is still a surrounded with uncertainties.

		\section{Motivation} \label{chap:introduction:sec:motivation}
		Devops represents a new way to look at the entire software pipeline. From development to delivery and maintenance, Devops represents an advantage for both teams and businesses by creating a more efficient, agile and collaborative way of working. DevOps also reduces the software time to market which provides a competitive advantage for those that are able to practice it.\\
		Being such a strong driver of positive changes, we believed that studying Devops, its values and common practices will enable broader adoption, further strengthening the movement and benefits for both practitioners and businesses.

		\section{Goals}
		The main goal of this thesis is to increase the existent knowledge regarding DevOps in order to enable teams and companies that want to adopt DevOps with the required understanding of common pitfalls and solutions related with DevOps. At the same time further studies of the DevOps movement will be able to build upon this study by extending the current concepts or by having a base from which to search new ones.

  	\section{Outline} \label{chap:introduction:sec:outline}
		This thesis documents a field study that tried to identify, near Portuguese startups, common practices and methodologies related with DevOps. \\
		Chapter \ref{chap:stateoftheart} introduces first the Cloud and then DevOps. This serves as, respectively, an introduction to some key concepts needed to understand this document and a base on which we based some of our study. The following sections, Patterns and a characterization of the Portuguese startup scene serve as justification for some of the choices described later in the document. \\
		Chapter \ref{chap:towardsdevops}, describes how the study was made including the methodology used, what information was extracted, the filters applied to achieve the final sample and how the extracted data was handled and compiled. \\
		Chapter \ref{chap:patterns} shows the results of the field study. In this chapter, thirteen patterns are described as well as the relations existing between them.\\
		Chapter \ref{chap:validation} shows the approach used to validate those patterns as well as the results of that validation.\\
		Finally, chapter \ref{chap:conclusion} identifies the main contributions of this thesis and suggests some future improvements.

\chapter*{Abstract}

Software projects have been increasing both in complexity and in size across the years.

As more businesses and opportunities arise, so did the challenges and the need for better and more robust practices and methodologies.
In 2001, with the writing of Manifesto for Agile Software Development and the subsequent movement we would see new ideas that tried to solve some of the known problems with software development. This new set of ideas and methodologies would, nevetheless do nothing to change the everlasting conflict between developers and operations.

As years passed and more companies moved online complexity grew and an extra pressure would be put both on the operations professionals and developers.
Attempts to heal this divide would eventually culminate in the appearance of DevOps.

Formed by combining the word Development with the word Operations, the "DevOps" is a movement that aims to change the way companies work. An amalgam of both cultura changes, new technologies and new practices, DevOps aims to change the way companies and teams structure themselves in order to promote a healthier work environment and at the same time increase the value delivered by teams.

Interpretations of the movement and what it means are still topics of discussion and written academic literature on the subject is still scarce.

In this article we study DevOps, first by using the existing literature as the basis to create a DevOps state-of-the-art and then by analyzing the current practices associated with DevOps in the real world.

The analysis, done by interviewing and observing 25 Portuguese startups is the basis from which we then extract a set of patterns related with DevOps and its values.

\chapter*{Abstract}

The DevOps movement aims to facilitate the work of both Developers and Operators by creating a new culture of collaboration between the two as well as introducing new practices and methodologies. \\
With initial discussions going back as far as 2008/2009, ideas regarding the DevOps movement and what it means are still evolving and, eight years later, are still topics of discussion even within the community. Combining that with the fact that some studies are showing a significant number of companies adopting DevOps and an equal amount looking into doing the same, it becomes worrying that DevOps literature is still largely based on personal opinions rather than scientific or academic literature. \\
This thesis represents a step toward a better and broader understanding, within the scientific community, of DevOps and its practices. With the objective of enabling more businesses to adopt DevOps and do so with less uncertainty we identify, in this thesis, 13 DevOps related patterns extracted from 25 Portuguese startup companies. Then, we conclude that the progress made with this thesis should be extended and we point out the direction to do so.



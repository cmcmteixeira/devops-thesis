\chapter*{Abstract}

The DevOps movement aims to reduce the articulation problems and the friction that exists between Developers and Operators by creating a new culture of collaboration between the two as well as introducing new practices and methodologies. \\
With initial developments going back as far as 2008/2009, ideas regarding the DevOps movement and what it means are still evolving and, eight years later, are still topics of discussion even within the community. Combining that with the fact that some studies are showing a significant number of companies adopting DevOps and an equal amount looking into doing the same it becomes worrying that DevOps literature is still largely based on personal opinions rather than scientific or academic literature.\\
With the objective of filling that vacuum, we looked into some of the Start-up companies operating in Portugal. We talked with 25 of them and tried to understand what challenges did they face and what solutions did they adopt to face them. With the knowledge gathered from this interviews we compiled 13 patterns that can serve as both an overview of some of the DevOps practices and as a basis on which future studies can build. Additionally we have also developed a DevOps state-of-the-art on which we define what is DevOps and what are the main ideals and goals that DevOps advocates.

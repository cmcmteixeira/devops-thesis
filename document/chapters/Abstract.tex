\chapter*{Abstract}

The DevOps movement aims to reduce the lack of articulation and the friction that exists between Developers and Operators. With that objective, the DevOps movement has shown to be able to reduce inefficiencies in both departments and the existent relation between the two. Companies that adopted DevOps were also able to decrease the time it took for new functionalities to reach the final customer and at the same time reduce the frequency of errors in that process.

Having been around since 2008/2009, ideas regardin the DevOps movement and what it means are still topics of discussion within the community and written academic literature on the subject is still scarce.

With the objective of improving and increasing the amount of knowledge regarding DevOps, we looked into some of the Startup companies operating in Portugal. With the knowledge gathered from interviewing 25 of those companies, we analyze and compile the extracted information into 13 patterns that can serve as both an overview of some of the DevOps practices and as a basis on which future studies can build.

\chapter*{Abstract}

Software and its development have been increasing both in complexity and in size. As more businesses were moving first to the Internet and then to the Cloud, new technologies and opportunities started to emerge. As the environment were they acted changed, most practices and mindesets stayed the same causing a imbalance between the external and internal environment of companies. In the midst of this confusion new ideas started to appear with the objective of restoring the balance between the environment and those who operated eventually culminating in the rise of DevOps.

But what is DevOps? Formed by combining the word Development with the word Operations, the word  "DevOps" has been around for sometime. Uses and appearances have been seen in different contexts and with different meanings making it a topic of discussion and discord. Being mostly refered to as a movement aiming to conciliate Software Operations and Software Developer professionals, there are those who see it as no more than a set of tools or a new job position. This mismatch of interpretations and overall lack of understanding of DevOps often means that companies and professionals lack the knowledge to fully take advantage of the benefits that DevOps brings. 

In this thesis we demystify and formalize DevOps, first by using the existing literature as a basis to create a DevOps state-of-the-art and then by analysing the current practices associated with DevOps in the real world. The analysis, done by interviewing and observing some startups from Portugal is then the basis from wich we then extract a set of patterns related with DevOps and its values. Some of the 13 identified patterns are then validated by watching the effects of the DevOps culture in real people in real world situations.

In the end we conclude that even though the collected information allows us to have some knoledge and grasp of DevOps and its broad scope, due to that same broad scope more work and investigation is needed in order to fully understand the movement and its consequences.
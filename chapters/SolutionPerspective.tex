\chapter{ Solution Perspective } \label{chap:solper}

	Collecting data from different companies will provide us with useful information on how to successfully adopt DevOps and what are the common pitfalls and traps, however, if not treated, this information can feel clumsy and disconnected. That is why, in addition to the collected data, common problems and solutions will be represented as patterns.  
    
    	\subsection{Why patterns?}
      	In the software development world there are several examples of different domains that were studied and where patterns were identified. Architecture, software development and Scrum are just some examples of that.
      
    	Patterns are by definition recurring and, in that sense, identifying them means identifying situations that will frequently happen. This allows us to prevent common pitfalls and be prepared with solutions in case we can not prevent them.
    
   	 	When identified, patterns allow us to have a common language while referring to domains of knowledge. Instead of having to constantly describe a situation or solution one can simply refer the pattern and, provided that there is a common knowledge of the pattern, communication can be greatly simplified.
    
    	Additionally patterns are also useful when reasoning about a subject. They allow us to abstract away the complexity of a situation and better visualize the overall solution as a set of pattern instead of an agglomerate of situations.
    
    	Lastly, although they are tied to a context, patterns are within that context generic, meaning that they are not tied to a specific moment in time or space. This makes them even more useful as they can be reused.
    
    	\subsection{Adjusting granularity}
    	In it's lowest level, granularity for the solutions presented could go as far as to specify what tools to use and how to configure and set them up. This would nonetheless be counterproductive as different companies have different requirements and different requirements require different tools. Consequently, the final solution will try to be tool agnostic and only present the situation and the types of solutions available with the hope to help the broader audience it can reach.

    	\subsection{Domains}
    
    	As DevOps touches different areas of knowledge both the cultural and technological aspects need to be considered. 
    
    	In the cultural side, aspects like how to redistribute operations teams or how to improve communication between peers will ,as well as other problems, be gathered and discussed. This will hopefully allow companies and their members to better accept the adoption of DevOps.
    
    	In the technological side, ways to improve the infrastructure elasticity, among others, will also be addressed.  



\chapter{State of The Art} \label{chap:stateoftheart}
    \section{The Internet}

    The Internet has existed since the 1970s yet, the vast majority of the current users only became familiar with it after 2005 \cite{Guangming2011}. Since then, the number of users of the Internet has been increasing and there are currently 3.2 billion people with Internet access representing 43\% of the total world population \cite{SANOU2015}.

    Economically, the size of the Internet is not less impressive.In 2010, in Europe, e-commerce represented around 3.5\% of all retail. Additionally, in the same study, around 40\% of individuals admitted to have made a purchase the year before they were asked \cite{EuropeanCommission2012}.

    In addition to this numbers, new developments in areas like IoT \footnote{Internet of Things} where devices will be able to interact with other devices \cite{Suresh2014} indicate that the already huge size and span of the Internet will only increase in the next years.


    \section{Cloud Computing} \label{chap:stateoftheart:sec:cloud}
    	If we had to pinpoint the start of the rise of popularity of the term \say{Cloud} we would probably refer to 2006 when Eric Schmidt, Google's CEO, used it to refer to the business model of providing services across the Internet\cite{Zhang2010}.  Since then the term was used in different contexts and with different meanings which lead to the lack of a agreed upon definition \cite{Zhang2010}.
    
    
    	\subsection{Definition} \label{chap:stateoftheheart:sec:cloud:sec:definition}
        Several definitions exist that try to capture the concept of cloud computing, in \say{A break in the cloud} \cite{Vaquero2008}, for instance, 22 different definitions of cloud are analyzed in an attempt to create a standard definition of \say{cloud}. 
        
        Nevertheless throughout this thesis we will use only one definition, namely, the one provided by NIST \footnote{National Institute of Standards and Technology of the U.S Department of Commerce}. NIST defines cloud computing as: 
        
        \say{Cloud computing is a model for enabling convenient, on-demand network access to a shared pool of configurable computing resources... that can be rapidly provisioned and released with minimal management effort or service provider interaction.}. 
        
        NIST, in the same article, also defines the essential characteristics of cloud computing to be:
        \begin{itemize}
			\item{On-Demmand self-service} - Cloud providers must allow their users to, without requiring human interaction, provision computing resources.
            \item{Broad network access} - Users must be allowed to access resources through standard mechanism. 
            \item{Resource pooling} - A multi-tenant model should be used in order to serve multiple users. Resources are allocated dynamically meaning that users do not know were, physically, the allocated resources are \cite{Garrison2012}.
            
            \item{Rapid elasticity} - Resources can be elastically allocated or deallocate. This should be possible to be done automatically \cite{Garrison2012,Mell2011}.
            
            \item{Measured service} - The usage of resources should be measured providing transparency in the provider-client relation.

		\end{itemize}
        
        \subsection{Delivery methods}
        
        In regards to their accessibility,it's common to identify three or four main categories of clouds \cite{Zhang2010,Garrison2012,Garrison2012,Khajeh-Hosseini2012} :
        	\begin{itemize}
            	\item{Public Clouds} - Public clouds are a pool of resources hosted by cloud providers who rent them to the general public. This resources can be accessed over the Internet and are shared among users. 
                
				\item{Private Clouds} - Private clouds are usually administered and used by the same organization. Alternatively a third party can also be hired to manage the resources. The main difference between public and private clouds is the usage of the resources. Private clouds resources are only used by one company as opposed to public clouds were resources are shared.

                \item{Hybrid Clouds} - Hybrid clouds combine both the private and public concepts. When using an hybrid clouds approach, infrastructure is divided by the two types of clouds meaning that some modules may be hosted in the private space and others on the public one.
                
                \item{Virtual Private Cloud} - Virtual private clouds are an alternative to private clouds. This type of cloud are essentially a public cloud that \textit{leverages virtual private network (VPN) technology..}\cite{Zhang2010} allowing users combine characteristics of both public clouds and private clouds. 
                
                Some authors choose not to mention VPCs \cite{Garrison2012,Garrison2012,Khajeh-Hosseini2012} while others do \cite{Zhang2010}. The reason why some do not mention it is mainly due to the fact that VPCs can be seen as a platform over a public cloud meaning that it is just a cloud. 
                
			\end{itemize}
        
        \subsection{Service Levels} \label{chap:stateoftheheart:sec:cloud:sec:servicelevels}
			In terms of service levels cloud computing can be classified in regards to the provided abstraction. The main categories are the following \cite{Garrison2012,Zhang2010,Mell2011,Sampaio2011,Vaquero2008} :
			\begin{itemize}
				\item{SaaS} - Software as a Service (SaaS) gives users access to a platform usually through a web client. Without the need to  download or install software the user is able to use the provided software instantly and virtually everywhere. 
	    
    			Applications of this model include messaging software, email services, collaborative platforms, etc.  
	    
   				\item{PaaS} - Platform as a Service (PaaS) allows it's users to quickly deploy applications with little to no configuration. In this type of platform environments are usually pre-setup or configurable. PaaS users should nevertheless expect only to be able to deploy applications or software supported by the provider.
	    
	    
    			\item{IaaS} - Infrastructure as a Service (IaaS) represents the lowest abstraction made available by cloud providers. In this model the user is able to configure and access a machine directly without constraints. This machine, usually a virtual server managed by the provider, can be configured and maintained by the user.
	    
    		This model is used when applications are complex and therefore need complex configurations.

			\end{itemize}

		\subsection{Benefits} \label{chap:stateoftheheart:sec:cloud:sec:benefits}
			The main advantages of cloud computing for it's users can be summarized as following:
			\begin{itemize}

				\item{Monetary Efficiency} - Cloud providers allow users to keep their resources to the needed minimum. By allowing users to quickly and easily increase/decrease the allocated resources amount and billing clients only for the resources used, cloud providers are good way to save money and spend only the needed amount \cite{Garrison2012,Mell2011}.
		    
				\item{Scalability} - usually through a public API of some kind most cloud providers allow for the quick increase or reduction of resources \cite{Mell2011}. This enables businesses to quickly go from zero to millions of users with minimum overhead. Additionally because processes related with the management and configuration of cloud servers can be automated it is usually possible to manage large systems with small teams \cite{Loukides2012}. 
		    
		    	\item{Maintainability} - Cloud Providers are responsible for the maintenance of all the hardware and infrastructure aspects. Cloud computing users therefore do not need to worry about updating the hardware or maintaining the physical infrastructure. This enables users to focus their resources in improving their product rather than improving the structure that supports it \cite{Garrison2012}.
			\end{itemize}
            
		\subsection{Challenges}

		The main challenges related with cloud computing are related with security aspects \cite{Zhang2010}. Given the remote accessibility nature of cloud computing resources, there is always fear that unwanted entities/parties may gain access to both the resources and the information that they hold. Some solutions to this problem include virtual private clouds or private clouds. 



	\section{DevOps}\label{sec:stateArtDevops}
		
        In \say{What is DevOps} \cite{Loukides2012}, Mike Loukides recalls a time when operations and development weren't separated and the same person that developed would also operate the equipment.  
        
        In the time between then and the present a great deal of things changed. Personal computers appeared, the internet grew to become a global network and two separate roles emerged in the software development life cycle: Developers and Operations. Developers, would rise as the creators of the software and operations, as the maintainers and managers of both the software and the infrastructure that supported it after it was developed.
        
        As maintainers of the software and infrastructure, operations would have ,as their goal, keeping both hardware and software working. This meant deploying new instances of software every time usage increased, adding or fixing servers and, when needed, upgrading the software and infrastructure. Manually done, in the beginning, and eventually automated with the help of scripts, this tasks would prove to be too hard and complex to be done efficiently and reliably. 
        
        New solutions to handle this problem had to be found and taking advantage of cloud computing and virtualized servers a shift from hardware owned resources to hosted resources begun.
        
        In the cloud, resources are virtualized and operations are no longer responsible for maintaining the hardware. Tools that allowed the configuration of virtualized resources in the form of code would soon appeared.
        
        	It became clear, as a result of the advances of cloud, that, the traditional role of operations no longer applied. As there was no hardware to manage and infrastructure could now be managed by code, the operational role had lost most of it's significance. The responsibilities that remain were, therefore, integrated in the development team. This team would now work closely with ex-operations to ensure the reliability of the produced software.   
	
      \subsection{Benefits}
      
      In \cite{Elliot2015}, a number of claims is made in regards to the possible benefits of adopting DevOps.
      These benefits are:
		\begin{itemize}
			\item{DevOps projects are believed to accelerate in 15\%-20\% the ability to delivery of capabilities to the client }
            
            \item{Adopting DevOps allows business to practice Continuous Delivery}
		\end{itemize}
      
      
      \subsection{Challenges}\label{devops:challenges}
    
    	DevOps is powerful in the sense that it allows companies to better react to different kind of scenarios and environments. Nevertheless the adoption of DevOps is not easy and a lot of companies struggle to achieve it. 
    
        Making the matters worst study or information relative to DevOps is scarce \cite{SaugatuckTechnology2014} and, the existent information, is usually spread throughout several sources. 

        Recently, in a survey made by \cite{SaugatuckTechnology2014} where around 300 development and IT companies were surveyed, it was asked what were the main issues identified when adopting DevOps. Some of the highlighted aspects emphasized  were: 
        \begin{itemize}
       	    \item{Overcoming cultural habits inside the organization.}
            \item{Lack of experience or understanding of DevOps practices.}
            \item{Lack of buy-in from leadership.}
    	\end{itemize}
        
        In another study, \cite{Debois2008} where three attempts to implement ideas similar to the ones defended by DevOps the same aspects related to cultural aspects as well as the lack of information are mentioned.
        	
        Both studies emphasize that adopting DevOps is far from being easy and that more work must be made in order to allow future DevOps adopters to fully understand the consequences,problems and solutions associated with DevOps.
      


	\section{The Portuguese startup scene}
    	Motivated by a recent investment in innovation and entrepreneurship, Portugal startups have been growing their position in the global startup scene \cite{Coleman2015}.  
        
       A study from 2015 \citet{StartupEuropePartnership2015} in which the Portuguese startup scene was analyzed, revealed that there were already 40 technology scaleups \footnote{Scaleups are companies that raised more than \$1M funding (since foundation) and had at least one funding event in the last five-year period } operating in Portugal at the time. The same study stated that this startups were able to raise a large portions of the received investment from international investors indicating, therefore, that the reach and scale of this startups was broader than just national arena. Additionally, it is also indicated in the study that Porto and Lisbon are the main centers of innovation, encompassing 70\% of the total of existing scaleups. In addition to the scaleups identified other smaller scale startups exist. Some of this startups are currently being incubated in incubators around the country. UPTEC \footnote{Science and Technology Park of University of Porto} and Startup Lisboa, both business incubators, have currently more than 300 companies \cite{Uptec,StartupLisboa} under their wing. 

\chapter{Introduction} \label{chap:introduction}

	\section{Context} \label{chap:introduction:sec:context}

		As the Internet grew and more people joined, an ever increasing number of
	businesses and opportunities started to emerge.
		
	    Initially, businesses that wanted to provide services over the Internet would need to buy servers, hire operations crews and buy real estate to accommodate the previous two. Additionally, businesses were also subjected to large usage fluctuations. This meant that a lot of the time, servers and teams were not being used in an optimal way. 
	    
	    Large upfront costs combined with large maintenance costs and unused resources were obviously not desirable and eventually solutions to this problems started to appear. Eventually named ”cloud computing”, the solution consisted in a new type of business that allowed other businesses to rent computing resources with little to no upfront costs. Cloud computing providers would charge businesses based on the usage as well as allow them to quickly increase or decrease computing resources. This elasticity combined with the elimination of the previously mentioned upfront costs meant that more businesses could be born and that the existing ones could lower their operational costs.

		In the midst of this change a new set of opportunities and problems started to arise. This ideas and problems eventually started the denominated DevOps movement. This movement called for the end of organizational boundaries between operations and development departments, integration of the previous two and the automation of repetitive tasks. This way DevOps aimed to reduce software time to market, teams sizes and also increase businesses agility and efficiency.	

	\section{Problem} \label{chap:introduction:sec:problem}

		The DevOps movement spans throughout a vast set of areas. This means that successfully understanding DevOps is not easy and can often seem like a never ending task requiring an informed and multidisciplinary approach.

		Regarding the current state of the literature on DevOps there is not a lot of investigation done on the subject \ref{devops:challenges}. 

		Companies that want do adopt DevOps are, as a result, required to learn the challenges and solutions of adopting DevOps while adopting it. Obviously this is far from being ideal and consequently companies may fail to adopt DevOps or practice it in an efficient way.

		   
	\section{Motivation \& Objectives} \label{chap:introduction:sec:motivation}

	    This work aims to provide companies and teams that want to adopt DevOps with knowledge relative to DevOps and the adoption process. This information should be easy to access and understand and, it should allow companies to have a feeling of the effort needed to adopt DevOps as well as allow them to define a preliminary road map for the adoption. 
	    
	    Lastly with the information provided, companies must be able to solve most of the problems that will appear when adopting DevOps.
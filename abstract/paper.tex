%"runningheads" enables:
%  - page number on page 2 onwards
%  - title/authors on even/odd pages
%This is good for other readers to enable proper archiving among other papers and pointing to content.
%Even if the title page states the title, when printed and stored in a folder, when blindly opening the folder, one could hit not the title page, but an arbitrary page. Therefore, it is good to have title printed on the pages, too.
\documentclass[runningheads,a4paper]{llncs}

%Even though `american`, `english` and `USenglish` are synonyms for babel package (according to https://tex.stackexchange.com/questions/12775/babel-english-american-usenglish), the llncs document class is prepared to avoid the overriding of certain names (such as "Abstract." -> "Abstract" or "Fig." -> "Figure") when using `english`, but not when using the other 2.
\usepackage[english]{babel}

%better font, similar to the default springer font
%cfr-lm is preferred over lmodern. Reasoning at http://tex.stackexchange.com/a/247543/9075
\usepackage[%
rm={oldstyle=false,proportional=true},%
sf={oldstyle=false,proportional=true},%
tt={oldstyle=false,proportional=true,variable=true},%
qt=false%
]{cfr-lm}
%
%if more space is needed, exchange cfr-lm by mathptmx
%\usepackage{mathptmx}

\usepackage{graphicx}

%extended enumerate, such as \begin{compactenum}
\usepackage{paralist}

%put figures inside a text
%\usepackage{picins}
%use
%\piccaptioninside
%\piccaption{...}
%\parpic[r]{\includegraphics ...}
%Text...

%Sorts the citations in the brackets
%\usepackage{cite}

\usepackage[T1]{fontenc}

%for demonstration purposes only
\usepackage[math]{blindtext}

%for easy quotations: \enquote{text}
\usepackage{csquotes}

%enable margin kerning
\usepackage{microtype}

%tweak \url{...}
\usepackage{url}
%nicer // - solution by http://tex.stackexchange.com/a/98470/9075
\makeatletter
\def\Url@twoslashes{\mathchar`\/\@ifnextchar/{\kern-.2em}{}}
\g@addto@macro\UrlSpecials{\do\/{\Url@twoslashes}}
\makeatother
\urlstyle{same}
%improve wrapping of URLs - hint by http://tex.stackexchange.com/a/10419/9075
\makeatletter
\g@addto@macro{\UrlBreaks}{\UrlOrds}
\makeatother

%diagonal lines in a table - http://tex.stackexchange.com/questions/17745/diagonal-lines-in-table-cell
%slashbox is not available in texlive (due to licensing) and also gives bad results. This, we use diagbox
%\usepackage{diagbox}

%required for pdfcomment later
\usepackage{xcolor}

% new packages BEFORE hyperref
% See also http://tex.stackexchange.com/questions/1863/which-packages-should-be-loaded-after-hyperref-instead-of-before

%enable hyperref without colors and without bookmarks
\usepackage[
%pdfauthor={},
%pdfsubject={},
%pdftitle={},
%pdfkeywords={},
bookmarks=false,
breaklinks=true,
colorlinks=true,
linkcolor=black,
citecolor=black,
urlcolor=black,
%pdfstartpage=19,
pdfpagelayout=SinglePage,
pdfstartview=Fit
]{hyperref}
%enables correct jumping to figures when referencing
\usepackage[all]{hypcap}

%enable nice comments
\usepackage{pdfcomment}
\newcommand{\commentontext}[2]{\colorbox{yellow!60}{#1}\pdfcomment[color={0.234 0.867 0.211},hoffset=-6pt,voffset=10pt,opacity=0.5]{#2}}
\newcommand{\commentatside}[1]{\pdfcomment[color={0.045 0.278 0.643},icon=Note]{#1}}

%compatibality with TODO package
\newcommand{\todo}[1]{\commentatside{#1}}

%enable \cref{...} and \Cref{...} instead of \ref: Type of reference included in the link
\usepackage[capitalise,nameinlink]{cleveref}
%Nice formats for \cref
\crefname{section}{Sect.}{Sect.}
\Crefname{section}{Section}{Sections}

\usepackage{xspace}
%\newcommand{\eg}{e.\,g.\xspace}
%\newcommand{\ie}{i.\,e.\xspace}
\newcommand{\eg}{e.\,g.,\ }
\newcommand{\ie}{i.\,e.,\ }

%introduce \powerset - hint by http://matheplanet.com/matheplanet/nuke/html/viewtopic.php?topic=136492&post_id=997377
\DeclareFontFamily{U}{MnSymbolC}{}
\DeclareSymbolFont{MnSyC}{U}{MnSymbolC}{m}{n}
\DeclareFontShape{U}{MnSymbolC}{m}{n}{
    <-6>  MnSymbolC5
   <6-7>  MnSymbolC6
   <7-8>  MnSymbolC7
   <8-9>  MnSymbolC8
   <9-10> MnSymbolC9
  <10-12> MnSymbolC10
  <12->   MnSymbolC12%
}{}
\DeclareMathSymbol{\powerset}{\mathord}{MnSyC}{180}

% correct bad hyphenation here
\hyphenation{op-tical net-works semi-conduc-tor}

\begin{document}

%Works on MiKTeX only
%hint by http://goemonx.blogspot.de/2012/01/pdflatex-ligaturen-und-copynpaste.html
%also http://tex.stackexchange.com/questions/4397/make-ligatures-in-linux-libertine-copyable-and-searchable
%This allows a copy'n'paste of the text from the paper
\input glyphtounicode.tex
\pdfgentounicode=1

\title{Towards DevOps: Practices and Patterns from the Portuguese Startup Scene}
%If Title is too long, use \titlerunning
%\titlerunning{Short Title}

%Single insitute
\author{Carlos Teixeira}
\author{Carlos Teixeira}
\author{Carlos Teixeira}
%If there are too many authors, use \authorrunning
%\authorrunning{First Author et al.}
\institute{Faculty of Engineering of the University of Porto}

%Multiple insitutes
%Currently disabled
%
			
\maketitle

\begin{abstract}

Software and its development have been increasing both in complexity and in size. As more businesses were moving first to the Internet and then to the Cloud, new technologies and opportunities started to emerge. 

Even though business environments were changing, most practices and mindsets stayed the same causing an imbalance between them and the requirements imposed by the environment. In the midst of this confusion new ideas started to appear with the objective of restoring the balance between the environment and those who operated in it. This ideas would eventually culminate in the rise of DevOps.

But what is DevOps? Formed by combining the word Development with the word Operations, the word  "DevOps" has been around for sometime. Uses and appearances have been seen in different contexts and with different meanings making it a topic of discussion and discord. Being mostly referred to as a movement aiming to conciliate Software Operations and Software Developer professionals, there are those who see it as no more than a set of tools or a new job position. This mismatch of interpretations and overall lack of understanding of DevOps often means that companies and professionals lack the knowledge to fully take advantage of the benefits that DevOps brings. 

In this article we demystify and formalize DevOps, first by using the existing literature as a basis to create a DevOps state-of-the-art and then by analyzing the current practices associated with DevOps in the real world. The analysis, done by interviewing and observing 25 Portuguese startups is then the basis from which we then extract a set of patterns related with DevOps and its values. Some of the 13 identified patterns are then validated by watching the effects of the DevOps culture in real people in real world situations.

\end{abstract}

\pagebreak
\section{Introduction}

DevOps is still a recent fenomena. The term is relatively new and was not yet traced back to a single moment in time. Nevertheless, a talk by Patrick Debois named "Agile Operations" is usually refered as the "starting point" of DevOps.
Before dwelving delving into what is DevOps it is important to understand the problem it tries to solve and what parties does it concern.

\subsection{The intervinients}

Operations were the maintainers and manageres of the final product. As maintainers , operations would have as their goal to keep both hardware and software working, deploying new instances of software every time usage increased, adding or fixing servers and, when needed, upgrading the software and infrastructure.

Developers on the other hand are the creators of the software/product. While concerns existed about code test coverage, code quality and code performance, usually developers would not fully understand how their product would be deployed.
As responsabilities go, developers would have to deliver increments to the product/software.

\subsection{The troubled relation}

While originally the Developer and Operations role were taken by same person, with time, the two roles started to diverge. 

As time went on, the operations professionals and the developers became increasingly specialized due to the increasing complexity of the tasks they had to perform. Waterfall like models would also reinforce this trend by separating the maintenance from the inplementation. Two departments would then start to appear creating a wide gap between developers and operations. 

Interactions would usually flow from the developers that developed new pieces of software to operations that would have to run and maintain it.

Objectives for developers would be to deliver new functionalities while operations would be incentivized to keep the application healthy.

Because their goal was to keep the application stable, operations would have little incentive to change in order to accomodate new changes and features delivered by developers. Performing this changes/updates would be riscky as new errors and problems could ocurr. Operations were therefore perversely incentivizes to make as little changes as possible.

Parallel to this situation, the siliozation of the Development and Operations departments meant that communication would seldom occur and would usually be in the form of instalation manuals and documentation that the operations team would receive from developers.

This would eventually lead to both a growing conflict between both departments an a sentiment of mistrust.

\subsection{The birth of DevOps}
As time passed the problem grew bigger and ideas and movements started to evolve that tried to reconcile this two departments, eventually culminating in the appearance of DevOps.

DevOps is therefore at its core an attempt to pacify the ongoing war between developers and operations professionals. 

In order to do so, some believed that taking advantage of newer technologies and tools like the Cloud and the automation possibilities that it brough would be enough. 

Others believed there was a need for more than just the introduction of newer tools and that a cultural change was needed. This cultural change would call for the end of the segregation between the two departments and for the inclusion of the Operations professionals into the development teams.

\subsection{DevOps values}

When looking at DevOps there are some key values and principles that should be taken into account. 

The first set of such values are the "Agile Manifesto Movement" ones (some authors go as far as to describe DevOps as an extension of the Agile Manifesto and Agile Methods to opearations and development). This values, sumed up in the manifesto as follows:
	\begin{itemize}
		\item Individuals and interactions over processes and tools
		\item Working software over comprehensive documentation
		\item Customer collaboration over contract negotiation
		\item Responding to change over following a plan
	\end{itemize}	

The second set of values is the John Willies CAMS acronym. CAMS stands for:
	\begin{itemize}
	\item Culture
	\item Automation
	\item Measurement
	\item Sharing 
	\end{itemize}
Together both this sets of values aim to improve the relation between developers and operations.


\section{Problem}

When searching for DevOps and related information, one could easily get lost by the ammount of information and the variety that exists. From job offers, to DevOps ready tools or cultural changes it is easy to get lost in the vasteness of existent information. Adding to this , the existent information is higly contraditory and both opinions/definitions often point in different directions. Contrastingly, information about DevOps in the academic world seems to be quite scarce existing only a few sources. 

All together, both the lack of concrete and verified information and the large ammount of different views means that DevOps is still misunderstood and that teams/companies that want to adopt it still do not know what are the common challenges and solutions available.


\section{Towards DevOps}

Given the low volume of information present in the current literature and the ammount of confusion in the community we had to base the study of DevOps on something more concret. With that in mind we decided that the way to go would be to look at were DevOps was being applied. This approach led us to choose the Portuguese Startups as a sample from which to extract information.

The choice of portuguese startups came from the following reasons:
	\begin{itemize}	
		\item Geographical proximity
		\item Easy identification: Most startups are concentrated in innovation centers and it is quite easy to identify and find startups.
		\item Resource scarcity: Usually startups have at most a couple tens of employees. This allows for a easier study of the overall companie and at the same time means that there is not a lot of room for specialized professionals or even departments. Additionally this also means 
		\item Role overload: Having few members usually translates in the need for members to adopt several roles.
		\item Culture: Due to "Resource scarcity" and "Role overload" startups usually have to adopt a more comprehensive approach and usually have a more holistic and collaborative approach for software development.
	\end{itemize}
Having identified the target there was a need to now define the methodology to extract information from these startups. 

The first option was to send a survey to every startup. This approach was nevertheless abandoned due to the following reasons:
	\begin{itemize}	
		\item People are not very receptive to fill long forms even if they are only multiple choice.
		\item DevOps is an extremely vast subject and trying to create a survey that identified every single detail and aspect of it would be non trivial.
		\item Surveys would not be very good at capturing the environment of companies.
	\end{itemize}
The second and chosen option was a little different. 

First we started by trying do identify what where the main areas that DevOps covers. 

We began by looking at "Culture". At the cultural side, devops defends communication, knowledge sharing and a multidisciplinar and holistic approach to software development. In order to identify how to achieve this goal/value we started by looking at how teams were setup. From team sizes, to team composition to cross team comunication and work management tools we aimed to caracterize this aspects. 

Secondly we dived into the Automation part of DevOps. In this area we looked at how environments were setup, how were pipelines structured, what software practices were put in place, how were infrastructure and deployments setup and what kind of tools were used.

Lastly we looked at measurement and attempt to understand what metrics companies were extracting and what responses would they give in different scenarios.

After identifying the areas we wanted to study we built a small script to guide the interview. Altough some of the areas overlaped (automation and measurement would usually be tighlty coupled for instance) we organized the information into six big areas :
	\begin{itemize}
	\item Product 
	\item Teams
	\item Pipeline
	\item Deployment
	\item Infrastructure
	\item Monitoring and error handling
	\end{itemize}
We attempted with this organization to be able to extract the information needed.

The next phase consisted of selecting, contacting and interviewing companies.For the selection phase we started by creating a list of portuguese startups (around 300 companies). We then removed those that did not develop software and those that seemed not to exist anymore (those that did not have online websites). By doing so we were able to cut the sample in half and 155 companies remained.

We then proceed to rank the remaining 155 companies. In order to rank them we created five metrics:
	\begin{itemize}
	
	\item Cloud Usage: This metric tried to estalish if a company used the cloud for its services.
	\item Offers SaaS or PaaS: We believed that if a company offered some kind of PaaS or PaaS then special considerations should exist in terms of scalling and monitoring.
	\item Number of employees: By looking at the companies website or linkedin profiles we were able to aproximately determine the number of employees of a company. Here we created three groups (Less than 5 members, Between 5 and 15 and more than 15)
	\item Reach: Companies that worked in a global scale would feel a bigger pressure to automate their infrastructure and services.
	\item Subjective appreaciation: While searching for the information specified above we identified some metrics that we felt mattered but would be difficult to parametrize. Such metrics would be for instance the fact that a question had only one or two employees with non technological backgrounds. In order to take that into account we added a subjective factor was added with values from 1 to 5.
	\end{itemize}

With this somehow arbitrary evaluation of companies we proceed to contact some of the companies in the list. In the end we contacted a total of 60 companies from those we were able to get a total of 25 interviews.


\end{document}
